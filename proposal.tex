\documentclass[12pt]{gatech-thesis}
\usepackage{amsmath,amssymb,latexsym,float,epsfig,subfigure}

%%
%% This example is adapted from ucthesis.tex, a part of the
%% UCTHESIS class package...
%%
\title{The Elements of Thesis} %% If you want to specify a linebreak
                               %% in the thesis title, you MUST use
                               %% \protect\\ instead of \\, as \\ is a
                               %% fragile command that \MakeUpperCase
                               %% will break!
\author{Perry H. Disdainful}
\department{School of Industrial and Systems Engineering}

%% Can have up to six readers, plus principaladvisor and
%% committeechair. All have the form
%%
%%  \reader{Name}[Department][Institution]
%%
%% The second and third arguments are optional, but if you wish to
%% supply the third, you must supply the second. Department defaults
%% to the department defined above and Institution defaults to Georgia
%% Institute of Technology.

\principaladvisor{Professor Alpha Betic}
\committeechair{Professor Ignatius Arrogant}
\firstreader{Professor General Reference}[School of Mathematics]
\secondreader{Professor Ivory Insular}[Department of Computer Science and Operations Research][North Dakota State University]
\thirdreader{Professor Earl Grey}
\fourthreader{Professor John Smith}
\fifthreader{Professor Jane Doe}[Another Department With a Long Name][Another Institution]
%\setcounter{secnumdepth}{2}
\degree{Doctor of Philosophy}

%% Set \listmajortrue below, then uncomment and set this for
%% interdisciplinary PhD programs so that the title page says
%% ``[degree] in [major]'' and puts the department at the bottom of
%% the page, rather than saying ``[degree] in the [department]''

%% \major{Algorithms, Combinatorics, and Optimization} 

\copyrightyear{2010}
\submitdate{August 2010} % Must be the month and year of graduation,
                         % not thesis approval! As of 2010, this means
                         % this text must be May, August, or December
                         % followed by the year.

%% The date the last committee member signs the thesis form. Printed
%% on the approval page.
\approveddate{1 July 2010}

%% \bibfiles{example-thesis}
\bibfiles{proposal}

%% The following are the defaults
%%    \titlepagetrue
%%    \signaturepagetrue
%%    \copyrightfalse
%%    \figurespagetrue
%%    \tablespagetrue
%%    \contentspagetrue
%%    \dedicationheadingfalse
%%    \bibpagetrue
%%    \thesisproposalfalse
%%    \strictmarginstrue
%%    \dissertationfalse
%%    \listmajorfalse
%%    \multivolumefalse

\begin{document}
\bibliographystyle{gatech-thesis}
%%
\begin{preliminary}
\begin{dedication}
\null\vfil
{\large
\begin{center}
To myself,\\\vspace{12pt}
Perry H. Disdainful,\\\vspace{12pt}
the only person worthy of my company.
\end{center}}
\vfil\null
\end{dedication}
\begin{preface}
Theses have elements.  Isn't that nice?
\end{preface}
\begin{acknowledgements}
I want to ``thank'' my committee, without whose ridiculous demands, I
would have graduated so, so, very much faster.
\end{acknowledgements}
% print table of contents, figures and tables here.
\contents
% if you need a "List of Symbols or Abbreviations" look into
% gatech-thesis-gloss.sty.
\begin{summary}
Why should I provide a summary?  Just read the thesis.
\end{summary}
\end{preliminary}
%%
\chapter{Introduction}

Every dissertation should have an introduction.  You might not realize
it, but the introduction should introduce the concepts, backgrouand,
and goals of the dissertation.

\section{Concepts}

This is where we talk about the concepts behind the dissertation.

\subsection{Primary Concept}

This is the primary concept.

\subsection{Secondary Concept}

This is the secondary concept.

\subsubsection{Even more secondary}

This is really not all that important.

\begin{table}
\caption{A table, centered.}
\begin{center}
\begin{tabular}{|l|r|}
  \hline 
Title & Author \\
\hline
War And Peace & Leo Tolstoy \\
The Great Gatsby & F. Scott Fitzgerald \\ \hline
\end{tabular}
\end{center}
\end{table}
%%
\chapter{Previous Work}

Some other research was once performed.

\begin{figure}
\caption{A first figure.}
\end{figure}

\begin{figure}
\caption{A second figure.}
\end{figure}
%%
\chapter{Conclusion}

\nocite{*}
%% We need this since this file doesn't ACTUALLY \cite anything...
%%
\appendix
\chapter{Some Ancillary Stuff}

Ancillary material should be put in appendices, which 
appear just before the bibliography. 

\begin{postliminary}
\references
\postfacesection{Index}{%
%%             ... generate an index here
%%         look into gatech-thesis-index.sty
}
\begin{vita}
Perry H. Disdainful was born in an insignificant town
whose only claim to fame is that it produced such a fine
specimen of a researcher.
\end{vita}
\end{postliminary}

\begin{abstract}
  This is the abstract that must be turned in as hard copy to the
  thesis office to meet the UMI requirements. It should \emph{not} be
  included when submitting your ETD. Comment out the abstract
  environment before submitting. It is recommended that you simply
  copy and paste the text you put in the summary environment into this
  environment. The title, your name, the page count, and your
  advisor's name will all be generated automatically.
\end{abstract}

\end{document}
