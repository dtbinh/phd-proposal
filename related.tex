\chapter{Related Works}

\section{Control of Highly Dynamic Motions}

Designing controllers for physically simulated biped characters
is a challenging problem due to its nonlinear dynamics and
under-actuated control.
Hodgins \etal \cite{Hodgins:1995:AHA,Wooten:1998:Phd} showed that 
carefully crafted control algorithms can simulate highly athletic 
motions, including diving, tumbling, vaulting, and leaping. 
Faloutsos \etal \cite{Faloutsos:2001:CCF} composed primitive 
controllers to simulate more complex motor skills, 
such as a kip-up move or a dive down stairs. 
Liu \etal \cite{Liu:2010:SCM} successfully tracked
contact-rich mocap sequences using a sampling-based approach. 
They showed that vigorous motions with complex contacts, such as a
dive-roll or a kip-up move, can be dynamically simulated, provided
full body mocap sequences as desired trajectories. 
Other techniques directly edit ballistic motion sequences
under the constraints imposed by conservation of momentum
\cite{Majkowska:2007:FPM,Sok:2010:EDH}, or apply a hybrid method for
synthesizing dynamic response to perturbation in the environment
\cite{Shapiro:2003:HCI}.  If the contact positions and timing are
known, spacetime optimization techniques can also generate compelling
dynamic motions
\cite{Liu:2002:SCD,Fang:2003:ESP,Safonova:2004:SPR,Sulejmanpavic:2004:APB}.
In this work, we take the approach of physical simulation, but we seek
for a more general and robust control algorithm such that the
controller can operate under a wide range of initial conditions and
allow for runtime perturbations. Furthermore, our controller does not
depend on any pre-scripted or captured motion trajectories.

\section{Control of Safe Falling Motions}

Safe falling and landing for bipeds is a topic that
receives broad attention in many disciplines. Robotic researchers are
interested in safe falling from standing height for the purpose of
reducing damages on robots due to accidental falls. Previous work has
applied machine learning techniques to predict falling
\cite{Kalyanakrishnan:2011:LPH}, as well as using an abstract model to
control a safe fall
\cite{Fujiwara:2002:FMC,Fujiwara:2007:OPF,Yun:2009:SFH}. In contrast
to the related work in robotics, our work focuses on falls from higher
places. In those cases, control strategies during long airborne phase
become critical for safe landing. We draw inspiration from
kinesiology literature and sport practitioners. In particular, the
techniques developed in freerunning and parkour community are of
paramount importance for designing landing control algorithms capable
of handling arbitrary scenarios
\cite{Edwardes:2009:TPF,HLJ:2011:URL}. 

\section{Human-in-the-loop Optimization}

\section{Learning by Demonstration}


