\chapter{Introduction}

\indent

Performing highly dynamic motions with agility and grace has been
one of the greatest challenges in sports, computer animation, and robotics.
A wide variety of athletics, such as acrobatics or free running, demonstrate the
efficient and artistic movements that involve abrupt changes of momentum
and contacts.
Furthermore, these motor skills are transferred to virtual characters 
in animations and games to express the intention of designers
and react to user interactions.
Robotics, another application of dynamic controllers, also started 
to tackle the agile movements and demonstrated
running, jumping, and landing motions with real hardwares.
Despite the recent progress, 
learning dynamic motor skills still remains a very difficult
problem because it needs to execute the task with great agility, 
ensure safety, and demonstrate self-expression.

In fact, developing dynamic controllers for virtual characters and
real robots can be considered related problems,
which can benefit each other.
Since the control problems in two domains have shared properties,
such as non-linearity, high-dimension, and discontinuity,
an algorithm developed in one domain can be transferred 
to the other domain.
However, control of real robots is more constrained
due to the sensor uncertainty and hardware limitations, which 
usually require more robustness than control of virtual characters.
Therefore, developing an algorithm in virtual environment to prove its
full capability and transferring it to real hardwares would be
a promising research direction, which is adopted in this proposal.

In this proposal, we introduce a set of techniques to expedite
the learning process of dynamic controllers for various dynamic motor 
skills. 
Particularly, we are interested in the following two problems:
%% The rest of the sections are organized as follows:

\begin{itemize}
\item 
  \textbf{Optimization of Falling and Landing In}
  \\
  motions this proposal, we tackle the problem of controlling safe falling 
  and landing motion for virtual characters and robots, which is a fundamental
  motor skill because highly dynamic motions involve the abrupt changes
  of contacts and can cause huge damages on the body parts.
  While absorbing the shock at the impact, a successful landing controller 
  also should be able to maintain readiness for the next action by managing
  the momentum properly.
  For the virtual character, we introduce a fast and robust optimization 
  algorithm for controlling falling and landing motions of virtual
  characters from a wide rage of heights and initial speeds.
  while reducing joint stress.
  Further, we propose a safe falling algorithm for a robot using
  a simulation-based optimization algorithm to capture the complex
  changes of contacts during the falling motion, which endures larger 
  external perturbatations comparing to the existing methods.
\item 
  \textbf{Human-guided learning of dynamic motions}
  \\
  Also, we investigate human-guide learning frameworks for dynamic
  motor skills from user instructions or demonstrations.
  These systems utilize the user-provided informations to
  accumulate the knowledge on the tasks and derive
  an optimal policy that reproduces the demonstrated behaviors.
  Since the learning of optimal policies can be done by simply
  watching a demonstration of the task to be performed,
  the development of controllers becomes much easier 
  than manual design.
  In our prior work, we introduce an iterative training system for dynamic
  motor skills inspired by human coaching techniques, which uses 
  human-in-the-loop (HITL) optimization for interactive training.  
  Further, we propose to develop a framework for learning 
  dynamic motor skills of humanoid robots
  from both demonstrations and instructions.
  This framework uses instructions as supplemental
  materials to demonstrations for identifying the proper domain
  of learning.
  As a result, the learning process of dynamic motor skills 
  becomes more intuitive and interactive.

\end{itemize}

