\chapter{Introduction}

%% \cite{fitts:1967:hp}

Performing highly dynamic motions with agility and grace has been
one of the greatest challenges in sports, computer animation, and robotics.
A wide variety of athletics, such as acrobatics or free running, demonstrate the
efficient and artistic movements that involve abrupt changes of momentum
and contacts.
Furthermore, these motor skills are transferred to virtual characters 
in animations and games to express the intention of designers
and react to user interactions.
Robotics, another application of dynamic controllers, also started 
to tackle the agile movements and demonstrated
running, jumping, and landing motions with real hardwares.
Despite the recent progress, 
learning dynamic motor skills still remains a very difficult
problem because it needs to execute the task with great agility, 
ensure safety, and demonstrate self-expression.

In fact, developing dynamic controllers for virtual characters and
real robots can be considered related problems,
which can benefit each other.
Since the control problems in two domains have shared properties,
such as non-linearity, high-dimension, and discontinuity,
an algorithm developed in one domain can be transferred 
to the other domain.
However, control of real robots is more constrained
due to the sensor uncertainty and hardware limitations, which 
usually require more robustness than control of virtual characters.
Therefore, developing an algorithm in virtual environment to prove its
full capability and transferring it to real hardwares would be
a promising research direction, which is adopted in this proposal.

%% %% Related in intro
%% In the computer animation and robotics community, various categories
%% of algorithms have been applied to control virtual and real characters.
%% For generating a sequence of dynamic motion, two approaches have been
%% frequently applied to control problems: tracking the reference motion, or 
%% solving the space-time optimization problem under physics constraints.
%% Both methods demonstrated that a variety of motions can be achieved 
%% by solving the optimization problem which considers the entire sequence
%% of motions.
%% However, the optimization over the entire motion usually requires
%% a long optimization time and also falls short of abilities
%% for adapting the motion to new environments.
%% On the other hand, abstract model based controllers can be interactively
%% adapted to a wide range of situations by capturing the essential features
%% of the dynamic motion.
%% This approach shows the robust control over various motions,
%% such as walking, balancing, and falling, but hard to consider the
%% very detailed features such as the exact boundary of characters 
%% or a sequence of contacts.
%% A sampling-based optimization for the parameterized controllers has proven
%% effective for optimizing the motion within a realistic simulation environment,
%% but it also takes a long time to be optimized, especially when the target
%% motion is parameterized or concatenated.
%% Instead, we want to introduce faster algorithms for developing
%% dynamic controllers, which also can handle a wide range of situations.

In this proposal, we introduce a set of techniques to expedite
the learning process of dynamic controllers for various dynamic motor 
skills. 
%% We first introduce a natural and safe landing controller for
%% the characters and robots, which is essential for highly dynamic motions.
%% After that, we propose an interactive syste to design dynamic controllers
%% for humanoids. 
The rest of the sections are organized as follows:

%% \section{Organizations}

\begin{itemize}
\item 
  \textbf{Optimization of Falling and Landing motions}
  In this proposal, we tackle the problem of controlling safe falling 
  and landing motion for virtual characters and robots, which is a fundamental
  motor skill because highly dynamic motions involve the abrupt changes
  of contacts and can cause huge damages on the body parts.
  While absorbing the shock at the impact, a successful landing controller 
  also should be able to maintain readiness for the next action by managing
  the momentum properly.
  For the virtual character, we introduce a fast and robust optimization 
  algorithm for controlling falling and landing motions of virtual
  characters from a wide rage of heights and initial speeds.
  while reducing joint stress.
  Further, we propose a safe falling algorithm for a robot using
  a simulation-based optimization algorithm to capture the complex
  changes of contacts during the falling motion, which endures larger 
  external perturbatations comparing to the existing methods.
\item 
  \textbf{Human-guided training of dynamic motions}
  Also, we investigate human-guide learning frameworks for dynamic
  motor skills from user instructions or demonstrations.
  These systems utilize the user-provided informations to
  accumulate the knowledge on the state-action pairs and derive
  a policy that reproduces the demonstrated or described behaviors.
  Since the learing of optimal policies can be done by simply
  watching a demonstration of the task to be performed,
  the development of controllers becomes much easier 
  than manual design.
  In our prior work, we introduce an iterative training system for dynamic
  motor skills inspired by human coaching techniques, which uses 
  human-in-the-loop (HITL) optimization for interactive training.  
  Further, we propose to develop a Learning from demonstration (LfD)
  framework for learning  dynamic motor skills of humanoid robots 
  from both demonstrations and instructions, by following 
  the example trajectories in a user-suggested control domain.
  %% Also, we introduce an human-guided learning framework for dynamic motor
  %% skills inspired by human coaching techniques, which can be considered
  %% as a human-in-the-loop (HITL) optimization or learning from 
  %% demonstrations (LfD) paradigm.
  %% In this system, the user only needs to provide a primitive initial controller
  %% and high-level, human-readable instructions as if coaching a human trainee.
  %% The virtual character interprets the provided instructions,
  %% accumulate the knowledge from the human coach,
  %% and iteratively improves its motor skills by optimizing control parameters.
  %% We propose a new hierarchical representation of controllers, 
  %% the ``motor tree'' for interpreting the instructions and accumulating
  %% the knowledge on motions.
  %% In addition, we develop a new sampling-based optimization method,
  %% Covariance Matrix Adaptation with Classification (CMA-C), 
  %% which efficiently solves the constrained problem by estimating
  %% the infeasible region from bad samples.
  %% The system is further extended by considering the hardware limitations 
  %% and uncertainties to support training dynamic motions for robots.
  %% With the proposed system, we demonstrate the design process of
  %% comlex dynamic controllers including jumps, vaults, rolls, and flips.


\end{itemize}

