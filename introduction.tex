\chapter{Introduction}

Learning a dynamic motor skill, such as a precision jump in Parkour or
a flip in gymanstics usually requires an iterative training process
with interactive coaching and repetitive practices.
Even though details of the learning process remain unknown,
Pitts and Posner \cite{fitts:1967:hp} hypothesize the three stages
of the skill acquisition process: the cognitive stage, 
the associative stage, and the autonomous stage.
The cognitive stage is when the trainee gathers information about the new
skill or receives the feedback on the existing skills from the coach
via instructions.
In the associative stage, the learner translates the declarative knowledge
to functional movements after an unspecified amount of practice and few mistakes.
In the autonomous stage, the skill has become almost automatic or habitual 
so that it can be executed with minimal amount of efforts.
A key of this hypothesis is an iterative process between cognitive learning
and physical training.
With this intuition, we propose a human-in-the-loop optimization 
framework to design dynamic controllers for virtual characters,
which consists of the ``coaching'' stage for receiving user instructions 
and the ``practicing'' stage for optimizing control parameters.

In fact, the iterative training system can greatly simplify the design
process of dynamic controllers by exploiting the same mechanism as human.
Moreover, since our system incrementally develops the controller by
accumulating human knowledge, the existing controllers can be easily
adapted, extended, or concatenated for new situations.
However, several new research questions arise when we formalize the
elusive principles of human learning into mathematical models for
designing physics-based controllers, such as mapping the human instruction
to low-level control variables or representing the accumulated knowledge
on motor skills.
To resolve the issues, we introduce a new hierarchical representation 
of motor skills, the ``motor tree'', to simulate the phenomenon of
skill abstraction and accumulation. 
To increase the responsiveness, the motor tree is optimized at interactive
rates by exploiting the previous failure history or idle time of the system.

%% In this proposal, we introduce a framework and related algorithm to simulate the learning process of human.
