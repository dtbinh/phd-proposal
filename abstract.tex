\begin{summary}
  As the technology of computer animation and robotics advances,
  controlling highly dynamic motions
  has been a great milestone for both virtual and real humanoid characters.
  However, developing controllers for dynamic motor skills is still 
  a challenging problem, which usually requires substantial amount of 
  design effort and optimization time due to its complexity.
  In this proposal, we introduce a set of techniques for developing
  controllers for various dynamic motor skills, such as jumping, rolling, 
  vaulting, and landing, for virtual characters and real robots.

  First, we introduce new algorithms to generate falling and landing motions,
  which are essential motor skills to ensure the safety of humans and robots.
  Previously, we  developed an online algorithm to control falling and 
  landing motions of virtual characters from a wide range of heights 
  and initial speeds, which can potentailly cause huge damages.
  Inspired by the falling control of a virtual character, 
  we propose an optimization algorithm for multi-contact falling
  motions of a real robot for minimizing the damage at the impact.
  Unlike the existing techniques that usually consider the desired contacts 
  as invariant features, 
  our simulation-based optimization can examine complex changes
  of contacts which allows the robot to break its fall with a longer
  seqeunce of actions.
  As a result, our controller can protect the robot from a wider
  range of situations including stronger perturbations.


  Second, we propose human-guided learning frameworks for designing dynamic
  controllers in simulated and real environments.
  In our prior work, the user can provide a sequence of high-level instructions
  to iteratively train dynamic controllers of characters 
  as if coaching a human trainee.
  Inspired by the training process of virtual characters,
  we further propose a framework for learning dynamic motor skills
  of robots from user provided demonstrations and instructions.
  In this project, we hypothesize that learning motor skills in the control domain
  is more straight-forward than learning kinematic trajectories.
  To this end, we combine demonstrations with high-level instructions
  to identify the proper control domain for learning.
  By learning dynamic motor skills with the user inputs, we can easily develop
  controllers for various dynamic motor skills, such as rolling,
  cartwheel, and yoga-balancing.

  %% By incorporating the proposed techniques, we can produce highly-dynamic motions 
  %% of virtual characters and real robots.

  %% As technology of computer animation and robotics advances,
  %% controlling highly dynamic motions
  %% has been great milestones for both virtual and real humanoid characters.
  %% However, developing controllers for dynamic motor skills is yet a challenging
  %% problem which usually requires substantial amount of design efforts
  %% due to its complexity.
  %% In this dissertation, we introduce a set of techniques to expedite the
  %% process of controller design for various dynamic motor skills,
  %% such as jumping, rolling, vaulting, and landing.
  %% First, we develop an online algorithm to control falling and landing motions 
  %% from a wide range of heights and initial speeds,
  %% which is an essential motor skill for dynamic motions. 
  %% We demonstrate that simple momentum planning with the proxy model of inertia 
  %% and feedback-based rules can generate the natural landing motions
  %% without large stresses on the joints.
  %% %% To avoid the large stress on the joints, we plan momentum trajectories 
  %% %% using the approximated inertia model and simple feedback-based rules.
  %% Second, we propose an iterative training framework to develop dynamic 
  %% controllers for virtual characters under the guidance of a human coach,
  %% which greatly simplifies the design process.
  %% The user only needs to provide a primitive initial controller and high-level, 
  %% human-readable instructions as if coaching a human trainee.
  %% Then the virtual character interprets the provided instructions,
  %% accumulate the knowledge from the human coach,
  %% and iteratively improves its motor skills by optimizing control parameters.
  %% Further, the optimization process is accelerated by 
  %% exploiting the previous history of failures and 
  %% continuously solving a parameterized objective function.
  %% By incorporating the propose techniques, complex dynamic motor skills
  %% can be intuitively and interactively generated without any reference motion.


  %% %% Human-in-the-loop

  %% We propose a human-in-the-loop (HITL) system to develop dynamic
  %% controllers for virtual characters under the guidance of a human coach.
  %% The user only needs to provide a primitive initial controller and high-level, 
  %% human-readable instructions as if coaching a human trainee.
  %% The virtual character interprets the provided instructions,
  %% accumulate the knowledge from the human coach,
  %% and iteratively improves its motor skills by optimizing control parameters.
  %% %% We introduce the ``motor tree'' as a new representation of motor skills
  %% %% to provide a direct mapping between high-level instructions and
  %% %% low-level control variables.
  %% To facilitate the mapping between high-level instructions and
  %% control variables, we introduce a new representation of motor skills,
  %% the ``motor tree'' which hierarchically organizes the skills from the low-level
  %% motions to the complex ones.
  %% The hierarchical structure enables flexible re-assembly and 
  %% efficient re-optimization by preserving the invariant features
  %% of motor skills.
  %% Further, the optimization process is accelerated by several techniques
  %% such as utilizing the failed previous trials or 
  %% exploiting the idling time of optimizer.
  %% With the propose framework, the human coach can design complex dynamic controller
  %% for virtual characters intuitively and interactively.

\end{summary}
