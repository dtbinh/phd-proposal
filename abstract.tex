\begin{summary}

We introduce a set of techniques to develop 
dynamic controllers for highly dynamic motor skills of humanoid characters,
including jumping, rolling, vaulting, and landing.
First, we develop an online algorithm to control falling and landing motions 
from a wide range of heights and initial speeds, 
continuously roll on the ground, and get back on its feet.
To avoid the large stress on the joints, we plan momentum trajectories 
using the approximated inertia model and feedback-based contact forces.
Second, we propose an iterative training framework to train dynamic 
controllers for virtual characters under the guidance of a human coach.
The user only needs to provide a primitive initial controller and high-level, 
human-readable instructions as if coaching a human trainee.
The virtual character interprets the provided instructions,
accumulate the knowledge from the human coach,
and iteratively improves its motor skills by optimizing control parameters.
Further, the optimization process is accelerated by 
exploiting the previous history of failures and 
continuously solving a parameterized objective function.
By incorporating the propose techniques, complex dynamic motor skills
can be intuitively and interactively generated without any reference motion.


  %% %% Human-in-the-loop

  %% We propose a human-in-the-loop (HITL) system to develop dynamic
  %% controllers for virtual characters under the guidance of a human coach.
  %% The user only needs to provide a primitive initial controller and high-level, 
  %% human-readable instructions as if coaching a human trainee.
  %% The virtual character interprets the provided instructions,
  %% accumulate the knowledge from the human coach,
  %% and iteratively improves its motor skills by optimizing control parameters.
  %% %% We introduce the ``motor tree'' as a new representation of motor skills
  %% %% to provide a direct mapping between high-level instructions and
  %% %% low-level control variables.
  %% To facilitate the mapping between high-level instructions and
  %% control variables, we introduce a new representation of motor skills,
  %% the ``motor tree'' which hierarchically organizes the skills from the low-level
  %% motions to the complex ones.
  %% The hierarchical structure enables flexible re-assembly and 
  %% efficient re-optimization by preserving the invariant features
  %% of motor skills.
  %% Further, the optimization process is accelerated by several techniques
  %% such as utilizing the failed previous trials or 
  %% exploiting the idling time of optimizer.
  %% With the propose framework, the human coach can design complex dynamic controller
  %% for virtual characters intuitively and interactively.

\end{summary}
