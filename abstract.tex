\begin{summary}
  As technology of computer animation and robotics advances,
  controlling highly dynamic motions
  has been a great milestone for both virtual and real humanoid characters.
  However, developing controllers for dynamic motor skills is yet a challenging
  problem which usually requires substantial amount of design efforts and
  optimization time due to its complexity.
  %% In this dissertation, we introduce a set of techniques to expedite
  %% the optimization process of dynamic controllers for various dynamic motor 
  %% skills, such as jumping, rolling, vaulting, and landing.
  In this proposal, we introduce a set of technique for training various
  highly dynamic motor skills, such as jumping, rolling, vaulting, and landing,
  to virtual characters and real robots.

  First, we introduce new algorithms to generate falling and landing motions,
  which are essential motor skills to ensure the safety of human and robots.
  We propose an online algorithm to control falling and landing motions 
  of virtual characters from a wide range of heights and initial speeds,
  which can potentailly cause huge damages.
  With the suggested algorithm, we demonstrate that simple momentum planning 
  with the proxy model of inertia and feedback-based rules can generate 
  the natural landing motions without large stresses on the joints.
  Inspired by falling of a virtual character, 
  we develop an optimization algorithm for multi-contact falling
  motions of a real robot for minimizing the damage at the impact.
  Unlikely the existing techniques which usually consider the desired contacts 
  as invariant features, we searches over a sequence of contacts to find
  the best falling motion which can handle a wider range of situations. 

  %% We also show that the sampling-based optimization of dynamic controllers
  %% can be accelerated by our new technique,
  %% Covariance Matrix Adaptation with Classification (CMA-C), 
  %% which utilizes failed simulations to approximate an infeasible region.
  %% In additition, we introduce a new fast optimization algorithm 
  %% for solving a parameterized objective function.
  %% Inspired by Schmidt's Variability Practice Hypothesis, 
  %% our algorithm simultaneously solves a continuous set of optimizers 
  %% for parameterized motions at the interactive rate 
  %% by sharing the simulation samples across the parameterized goals.

  Second, we propose a human-in-the-loop (HITL) system to develop dynamic
  controllers under the guidance of a human coach.
  %% The user only needs to provide a primitive initial controller and high-level, 
  %% human-readable instructions as if coaching a human trainee.
  %% The virtual character interprets the provided instructions,
  %% accumulate the knowledge from the human coach,
  %% and iteratively improves its motor skills by optimizing control parameters.
  In this system, the user can provide a sequence of high-level instructions
  to iteratively train dynamic controllers of characters 
  as if coaching a human trainee.
  To facilitate the mapping between high-level instructions and
  control variables, we introduce a new representation of dynamic controllers,
  the ``motor tree'', which also enables flexible re-assembly and 
  efficient re-optimization by preserving the invariant features
  of motor skills.
  Further, the optimization process is accelerated by 
  utilizing the failed previous trials.
  Using our proposed system, we also train a real robot to perform
  dynamic motor skills such as rolling under enough consideration 
  on noises over control and simulation.

  By incorporating the propose techniques, we can produce highly-dynamic motions 
  of virtual characters and real robots.

  %% As technology of computer animation and robotics advances,
  %% controlling highly dynamic motions
  %% has been great milestones for both virtual and real humanoid characters.
  %% However, developing controllers for dynamic motor skills is yet a challenging
  %% problem which usually requires substantial amount of design efforts
  %% due to its complexity.
  %% In this dissertation, we introduce a set of techniques to expedite the
  %% process of controller design for various dynamic motor skills,
  %% such as jumping, rolling, vaulting, and landing.
  %% First, we develop an online algorithm to control falling and landing motions 
  %% from a wide range of heights and initial speeds,
  %% which is an essential motor skill for dynamic motions. 
  %% We demonstrate that simple momentum planning with the proxy model of inertia 
  %% and feedback-based rules can generate the natural landing motions
  %% without large stresses on the joints.
  %% %% To avoid the large stress on the joints, we plan momentum trajectories 
  %% %% using the approximated inertia model and simple feedback-based rules.
  %% Second, we propose an iterative training framework to develop dynamic 
  %% controllers for virtual characters under the guidance of a human coach,
  %% which greatly simplifies the design process.
  %% The user only needs to provide a primitive initial controller and high-level, 
  %% human-readable instructions as if coaching a human trainee.
  %% Then the virtual character interprets the provided instructions,
  %% accumulate the knowledge from the human coach,
  %% and iteratively improves its motor skills by optimizing control parameters.
  %% Further, the optimization process is accelerated by 
  %% exploiting the previous history of failures and 
  %% continuously solving a parameterized objective function.
  %% By incorporating the propose techniques, complex dynamic motor skills
  %% can be intuitively and interactively generated without any reference motion.


  %% %% Human-in-the-loop

  %% We propose a human-in-the-loop (HITL) system to develop dynamic
  %% controllers for virtual characters under the guidance of a human coach.
  %% The user only needs to provide a primitive initial controller and high-level, 
  %% human-readable instructions as if coaching a human trainee.
  %% The virtual character interprets the provided instructions,
  %% accumulate the knowledge from the human coach,
  %% and iteratively improves its motor skills by optimizing control parameters.
  %% %% We introduce the ``motor tree'' as a new representation of motor skills
  %% %% to provide a direct mapping between high-level instructions and
  %% %% low-level control variables.
  %% To facilitate the mapping between high-level instructions and
  %% control variables, we introduce a new representation of motor skills,
  %% the ``motor tree'' which hierarchically organizes the skills from the low-level
  %% motions to the complex ones.
  %% The hierarchical structure enables flexible re-assembly and 
  %% efficient re-optimization by preserving the invariant features
  %% of motor skills.
  %% Further, the optimization process is accelerated by several techniques
  %% such as utilizing the failed previous trials or 
  %% exploiting the idling time of optimizer.
  %% With the propose framework, the human coach can design complex dynamic controller
  %% for virtual characters intuitively and interactively.

\end{summary}
